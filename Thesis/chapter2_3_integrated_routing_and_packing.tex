\section{Integrated Routing and Packing}
\label{sec:integrated_routing_and_packing}

Although the Vehicle Routing Problem (VRP) has been studied for over sixty years, it continues to be extended in new directions. Many of these extensions introduce additional constraints or objectives, aiming to make the model more practical and applicable to modern logistics settings.

One of the most natural ways to extend the VRP is to add a constraint that addresses the demand of each customer, called the \textbf{Capacitated Vehicle Routing Problem (CVRP)}. In the literature, the terms VRP and CVRP are often used interchangeably \cite{braekers2016vehicle}, since the classical version of the problem almost always includes a vehicle capacity constraint; without it, there is no strong reason to use multiple vehicles.

The capacity mentioned is typically considered in terms of either weight or volume as a single value. A more realistic approach is to take into account the actual dimensions of deliveries—the resulting task being to find the physical arrangement of items in vehicles using either a Two- or Three-Dimensional Packing Problem.

An exact method that builds on this idea is proposed by Iori et al. \cite{iori2007exact}, who addresses the Vehicle Routing Problem with Two-Dimensional Loading Constraints. In their model, the demand of each customer is given as a list of items that must be packed directly into the vehicle. The packing must respect geometric feasibility—items cannot overlap, and orientation rules or loading constraints must be satisfied. Other extensions focus on route balancing—making sure the workload is fairly distributed between vehicles—or enforce rules on the order in which customers are visited. These additions make the model more realistic but also more difficult to solve.

\subsection{Two-Level Vehicle Routing and Loading Problem}

The models discussed so far assume that items are loaded directly into vehicles with different constraints, such as two- or three-dimensional geometry, route balance, or LIFO unloading rules. In most real distribution systems, however, there is at least one intermediate step: orders are first placed onto pallets, roll cages, or containers, and only then loaded into trucks. This motivates a family of \textbf{two-level} problems in which both the grouping of items into pallets and the assignment of pallets to vehicles and routes are part of the decision-making process.

Early work on this idea appears implicitly in pallet-based routing models. Zachariadis et al. introduce the \textbf{Pallet-Loading Vehicle Routing Problem} \cite{zachariadis2012}, where customer boxes are first packed onto pallets and then pallets are loaded into trucks under two-dimensional loading and routing constraints. Their subsequent works \cite{zachariadis2013a, zachariadis2013b} show that integrating pallet construction with routing can substantially reduce total cost compared to treating loading as a separate post-processing step, and they tackle the problem with sophisticated metaheuristics and compact neighbourhood structures. In parallel, Moura and co-authors study two-stage packing and loading procedures in industrial contexts \cite{moura2009, moura2017, moura2019, moura2023matheuristic}, combining three-dimensional loading rules with realistic side constraints such as time windows, simultaneous delivery and pickup, or restrictions on how products may be stacked. Their approaches are typically model-based or matheuristic: a mixed-integer formulation is used to capture the integrated structure, and heuristic search or decomposition is employed to handle larger instances.

Following the taxonomy introduced by Liu et al.\ \cite{liu2024integrated}, we refer to this class of problems as the \textbf{Two-Level Vehicle Routing and Loading Problem (2L-VRLP)}. In the 2L-VRLP, a set of items \(I\) must be delivered from a depot to a set of destinations \(D\). Each item \(i \in I\) has a known destination and size (volume or three-dimensional dimensions). Items are first packed onto pallets from a set \(J\), each pallet \(j \in J\) having a limited capacity and an associated fixed cost. The loaded pallets are then packed onto vehicles from a fleet \(K\), where each vehicle \(k \in K\) has its own capacity, fixed cost and must follow a feasible route on the transportation network defined over \(D\). Depending on the variant, pallets may be restricted to contain items for a single destination, for a cluster of nearby destinations, or freely mix destinations; similarly, loading constraints at pallet and vehicle level can range from one-dimensional volume limits to full three-dimensional placement with stability and unloading rules.

Formally, the 2L-VRLP integrates two bin packing stages and a capacitated VRP into a single optimisation model. Decisions include: (i) assigning each item to exactly one pallet and determining a feasible loading pattern on that pallet, (ii) assigning each pallet to exactly one vehicle and ensuring that pallets can be feasibly loaded into the vehicle, and (iii) constructing a route for each used vehicle so that all destinations associated with its pallets are visited exactly once. The objective is to minimise the total cost, typically defined as the sum of pallet and vehicle fixed costs plus routing costs on the underlying network. In the remainder of this section we adopt this definition as the reference framework for two-level routing and loading, and use it to position existing contributions as well as the specific variant addressed in this thesis.


\subsection{Solution approaches for integrated Packing and Routing problems}