\section{Vehicle Routing Problem}
\label{sec:vehicle_routing_problem_and_variants}

The \textbf{Vehicle Routing Problem (VRP)} is a fundamental problem in logistics and operations research. It involves designing a set of routes for vehicles that start and end at a depot and collectively serve a group of customers. The main goal is to minimise the total cost of transportation, often measured by distance or time, while respecting operational constraints such as vehicle capacity or maximum route duration.

Since its first formal introduction by Dantzig and Ramser in 1959 \cite{dantzig1959truck}, the VRP has become a core problem in combinatorial optimisation. It has many practical applications in distribution systems, and a large number of variants have been developed to reflect real-world conditions. Despite its relatively simple definition, solving the VRP is computationally difficult, and most real-world instances require the use of heuristics or metaheuristics. In this chapter, the classical VRP is introduced, followed by an overview of the most common approaches to solving it.

\subsection{Definition of the Vehicle Routing Problem}

The aim of the classical VRP is to find a set of routes, all starting and ending at a depot, for a fleet of vehicles, such that each customer is visited exactly once and the total routing cost is minimised.

Formally, let us consider an undirected graph \(G = (V, E)\), where \(V = \{v_0, v_1, \dots, v_n\}\) is the set of vertices and \(E = \{(v_i, v_j) : v_i, v_j \in V, i < j\}\) is the set of edges. The vertex \(v_0\) represents the depot, and the remaining vertices correspond to customer locations. Each edge \((v_i, v_j)\) has an associated non-negative cost \(c_{ij}\), typically representing distance or travel time. It is assumed that the cost matrix is symmetric. \cite{cordeau2002guide}

Each customer \(v_i\) has a demand \(d_i\), and each vehicle has a capacity limit \(Q\). The task is to construct at most \(m\) routes such that:
\begin{itemize}
    \item each route starts and ends at the depot,
    \item each customer is visited exactly once by one vehicle,
    \item the total demand on each route does not exceed the vehicle capacity,
    \item the total cost of all routes is minimised.
\end{itemize}

The VRP can be seen as a generalisation of the \textit{Travelling Salesman Problem (TSP)}. However, it is harder in practice due to additional constraints and the need to partition customers among multiple vehicles. Exact algorithms exist, but they are limited to relatively small instances—typically up to 100 customers \cite{laporte2009fifty}. Because of that, most practical VRPs are solved using heuristics or metaheuristics.

\newpage

\subsection{Solution Approaches for the Vehicle Routing Problem}

As in the previous chapter about the Bin Packing Problem, I have focused mostly on exact methods, only briefly mentioning the use of heuristics. In the case of the VRP, my attention will shift almost entirely toward heuristic approaches.

It is, however, important to mention that exact methods for solving the VRP exist. A short collection of them can be found in the 2009 survey \textit{Fifty Years of Vehicle Routing} \cite{laporte2009fifty}, including \textbf{Branch-and-Bound}, \textbf{Dynamic Programming}, and \textbf{Vehicle Flow} formulations, although none of them are able to solve instances with more than about 135 vertices. Similarly to the BPP, algorithms using \textbf{Set Partitioning}, \textbf{Column Generation}, or \textbf{Branch-and-Price} techniques perform slightly better but remain limited to medium-sized instances.

In heuristic algorithms, there is a distinction between \textbf{classical heuristics} and \textbf{metaheuristics}, based on their structure, flexibility, and generality. A useful comparison of both groups can be found in the survey \textit{A Guide to Vehicle Routing Heuristics} \cite{cordeau2002guide}.

The first classical heuristic mentioned both in \cite{laporte2009fifty} and \cite{cordeau2002guide} is the \textbf{Savings Algorithm}, proposed in 1964 by Clarke and Wright \cite{clarke1964scheduling}. It starts from an initial solution consisting of \(n\) back-and-forth routes and iteratively merges them to reduce cost. At each step, the algorithm selects a merge that results in the largest cost saving, defined as \(s_{ij} = c_{i0} + c_{0j} - c_{ij}\), provided the merge is feasible with respect to capacity and other constraints.

Another early constructive heuristic is the \textbf{Sweep Algorithm}, introduced by Gillett and Miller (1974) \cite{gillett1974heuristic}. It is particularly suited for instances where customers are distributed in a plane. The algorithm begins by computing the polar angle of each customer relative to the depot. Customers are then sorted by angle, and assigned to routes by sweeping a ray around the depot in angular order. As customers are added, a new route is started whenever the current one would exceed the vehicle's capacity or maximum route length. Once clusters are formed, each route is solved as a TSP. The sweep algorithm is simple and fast, and is often used to generate an initial solution for improvement heuristics.

Another well-known classical heuristic is the \textbf{Fisher and Jaikumar algorithm} \cite{fisher1981generalized}, which belongs to the family of \textbf{Cluster-First, Route-Second} methods. These heuristics work by first partitioning the set of customers into smaller groups (or clusters), and then solving a routing problem within each group separately. The goal is to reduce the complexity of the problem by breaking it into smaller, more manageable subproblems. In practice, the performance of this algorithm depends heavily on the selection of good cluster "seeds"—points that act as centres for forming the clusters; hence, this method tends to be less flexible and more difficult to implement than other classical heuristics. Nonetheless, it laid the groundwork for many later approaches that combine clustering and routing in different ways.

Compared to classical heuristics, \textbf{metaheuristics} are more general-purpose frameworks designed to explore the solution space more thoroughly. They often incorporate randomisation, memory, or adaptive strategies, which allow them to escape local optima and perform well across a wide range of problem instances.

One of the most common strategies used within metaheuristics is \textbf{local search}. These methods start from an initial solution and iteratively move to a neighbouring one by applying small changes, such as swapping, inserting, or removing customers from routes. The process continues as long as improvements can be found. Local search is often used as a building block within more advanced metaheuristics, and can also be applied as a post-optimisation step to refine solutions obtained by constructive heuristics.

\textbf{Simulated Annealing (SA)} is one of the earliest metaheuristics applied to the VRP. Inspired by the annealing process in metallurgy, it allows worse solutions to be accepted with a certain probability, which decreases over time. This mechanism helps the algorithm escape local optima in the early phases of the search, while gradually shifting towards intensification as the “temperature” cools \cite{kirkpatrick1983optimization}. Although simple, Simulated Annealing can be effective when combined with a well-designed neighbourhood structure and cooling schedule.

\textbf{Tabu Search (TS)} has been particularly influential in VRP research and applications. It is a local search method enhanced with memory: recently visited solutions (or moves) are placed on a “tabu list” to prevent the search from revisiting them too soon \cite{glover1990tabu}. This helps the algorithm explore more of the solution space and avoid getting stuck in cycles. Tabu Search typically incorporates aspiration criteria and strategic restarts, and it can be adapted to a wide range of VRP variants. It remains one of the most widely used metaheuristics for the problem \cite{golden1998metaheuristics}.

Several other metaheuristics have also been applied successfully to the VRP. \textbf{Genetic Algorithms (GA)} and \textbf{Ant Colony Optimisation (ACO)} are population-based methods that construct or evolve solutions through repeated iterations. GA applies operators inspired by natural selection—such as crossover and mutation—while ACO mimics the behaviour of ants finding paths by laying and following pheromone trails \cite{dorigo1997ant}. Both methods require careful parameter tuning and are often combined with local search to improve solution quality.

Another group of methods focuses on combining greedy construction with local improvement. \textbf{GRASP (Greedy Randomised Adaptive Search Procedure)} builds multiple solutions using a randomised greedy heuristic, followed by local search \cite{resende2010greedy}. \textbf{Variable Neighbourhood Search (VNS)} explores different neighbourhood structures in a systematic way, changing the scope of the search as needed \cite{mladenovic1997variable}. Both GRASP and VNS are relatively easy to implement and have been shown to perform well on a variety of VRP instances.

Finally, many practical approaches are based on \textbf{hybrid metaheuristics}, which combine features from multiple strategies. For example, a genetic algorithm might include a tabu-based local search as a refinement step, or a GRASP method may be used to initialise an ACO algorithm. These hybrids aim to balance intensification (searching near good solutions) and diversification (exploring new areas of the space), and are often among the most effective approaches in benchmark studies \cite{golden1998metaheuristics}.
